%%%%%%%%%%%%%%%%%%%%%%%%%%%%%%%%%%%%%%%%%
% Programming/Coding Assignment
% LaTeX Template
%
% This template has been downloaded from:
% http://www.latextemplates.com
%
% Original author:
% Ted Pavlic (http://www.tedpavlic.com)
%
% Note:
% The \lipsum[#] commands throughout this template generate dummy text
% to fill the template out. These commands should all be removed when 
% writing assignment content.
%
% This template uses a Perl script as an example snippet of code, most other
% languages are also usable. Configure them in the "CODE INCLUSION 
% CONFIGURATION" section.
%
% This template was modified by Edouard Klein (http://rdklein.fr)
% to be used for technical report edition
% while working for the Gendarmerie Nationale
%
%%%%%%%%%%%%%%%%%%%%%%%%%%%%%%%%%%%%%%%%%

%----------------------------------------------------------------------------------------
%	PACKAGES AND OTHER DOCUMENT CONFIGURATIONS
%----------------------------------------------------------------------------------------

\documentclass{article}

\usepackage[frenchb]{babel}
\usepackage[utf8]{inputenc}
\usepackage[T1]{fontenc}
\usepackage[hyphens]{url}
\usepackage{hyperref} 
\usepackage{fancyhdr} % Required for custom headers
\usepackage{lastpage} % Required to determine the last page for the footer
\usepackage{extramarks} % Required for headers and footers
\usepackage[usenames,dvipsnames]{color} % Required for custom colors
\usepackage{graphicx} % Required to insert images
\usepackage{listings} % Required for insertion of code
\usepackage{courier} % Required for the courier font
\usepackage{lipsum} % Used for inserting dummy 'Lorem ipsum' text into the template

% Margins
\topmargin=-0.45in
\evensidemargin=0in
\oddsidemargin=0in
\textwidth=6.5in
\textheight=9.0in
\headsep=0.25in

\linespread{1.1} % Line spacing

% Set up the header and footer
\pagestyle{fancy}
\lhead{\hmwkClass\ (\hmwkShortTitle\ \hmwkClassTime)} % Top center head
\rhead{\firstxmark} % Top right header
\lfoot{\lastxmark} % Bottom left footer
\cfoot{} % Bottom center footer
\rfoot{Page\ \thepage\ of\ \protect\pageref{LastPage}} % Bottom right footer
\renewcommand\headrulewidth{0.4pt} % Size of the header rule
\renewcommand\footrulewidth{0.4pt} % Size of the footer rule

\setlength\parindent{0pt} % Removes all indentation from paragraphs

%----------------------------------------------------------------------------------------
%	CODE INCLUSION CONFIGURATION
%----------------------------------------------------------------------------------------

\definecolor{MyDarkGreen}{rgb}{0.0,0.4,0.0} % This is the color used for comments
\lstloadlanguages{Python} % Load Python syntax for listings, for a list of other languages supported see: ftp://ftp.tex.ac.uk/tex-archive/macros/latex/contrib/listings/listings.pdf
\lstset{language=Python, % Use Python in this example
        frame=single, % Single frame around code
        basicstyle=\small\ttfamily, % Use small true type font
        keywordstyle=[1]\color{Blue}\bf, % Perl functions bold and blue
        keywordstyle=[2]\color{Purple}, % Perl function arguments purple
        keywordstyle=[3]\color{Blue}\underbar, % Custom functions underlined and blue
        identifierstyle=, % Nothing special about identifiers                                         
        commentstyle=\usefont{T1}{pcr}{m}{sl}\color{MyDarkGreen}\small, % Comments small dark green courier font
        stringstyle=\color{Purple}, % Strings are purple
        showstringspaces=false, % Don't put marks in string spaces
        tabsize=5, % 5 spaces per tab
        %
        % Put standard Perl functions not included in the default language here
        morekeywords={rand},
        %
        % Put Perl function parameters here
        morekeywords=[2]{on, off, interp},
        %
        % Put user defined functions here
        morekeywords=[3]{test},
       	%
        morecomment=[l][\color{Blue}]{...}, % Line continuation (...) like blue comment
        numbers=left, % Line numbers on left
        firstnumber=1, % Line numbers start with line 1
        numberstyle=\tiny\color{Blue}, % Line numbers are blue and small
        stepnumber=5 % Line numbers go in steps of 5
}

% Creates a new command to include a python script, the first parameter is the filename of the script (without .pl), the second parameter is the caption
\newcommand{\pythonscript}[2]{
\begin{itemize}
\item[]\lstinputlisting[caption=#2,label=#1]{#1.py}
\end{itemize}
}

%----------------------------------------------------------------------------------------
%	DOCUMENT STRUCTURE COMMANDS
%	Skip this unless you know what you're doing
%----------------------------------------------------------------------------------------

% Header and footer for when a page split occurs within a problem environment
\newcommand{\enterProblemHeader}[1]{
\nobreak\extramarks{#1}{#1 continue sur la page suivante\ldots}\nobreak
\nobreak\extramarks{#1 (suite)}{#1 t\ldots}\nobreak
}

% Header and footer for when a page split occurs between problem environments
\newcommand{\exitProblemHeader}[1]{
\nobreak\extramarks{#1 (suite)}{#1 continue sur la page suivante\ldots}\nobreak
\nobreak\extramarks{#1}{}\nobreak
}

\setcounter{secnumdepth}{0} % Removes default section numbers
\newcounter{homeworkProblemCounter} % Creates a counter to keep track of the number of problems

\newcommand{\homeworkProblemName}{}
\newenvironment{homeworkProblem}[1][Problem \arabic{homeworkProblemCounter}]{ % Makes a new environment called homeworkProblem which takes 1 argument (custom name) but the default is "Problem #"
\stepcounter{homeworkProblemCounter} % Increase counter for number of problems
\renewcommand{\homeworkProblemName}{#1} % Assign \homeworkProblemName the name of the problem
\section{\homeworkProblemName} % Make a section in the document with the custom problem count
\enterProblemHeader{\homeworkProblemName} % Header and footer within the environment
}{
\exitProblemHeader{\homeworkProblemName} % Header and footer after the environment
}

\newcommand{\problemAnswer}[1]{ % Defines the problem answer command with the content as the only argument
\noindent\framebox[\columnwidth][c]{\begin{minipage}{0.98\columnwidth}#1\end{minipage}} % Makes the box around the problem answer and puts the content inside
}

\newcommand{\homeworkSectionName}{}
\newenvironment{homeworkSection}[1]{ % New environment for sections within homework problems, takes 1 argument - the name of the section
\renewcommand{\homeworkSectionName}{#1} % Assign \homeworkSectionName to the name of the section from the environment argument
\subsection{\homeworkSectionName} % Make a subsection with the custom name of the subsection
\enterProblemHeader{\homeworkProblemName\ [\homeworkSectionName]} % Header and footer within the environment
}{
\enterProblemHeader{\homeworkProblemName} % Header and footer after the environment
}

%----------------------------------------------------------------------------------------
%	NAME AND CLASS SECTION
%----------------------------------------------------------------------------------------

\newcommand{\hmwkTitle}{Rapport d'analyse} % Assignment title
\newcommand{\hmwkDueDate}{\today} % Due date
\newcommand{\hmwkClass}{PJGN/SCRC/C3N} % Unité
\newcommand{\hmwkClassTime}{} % Vide
\newcommand{\hmwkClassInstructor}{Disque dur infecté par un méchant virus} % Objet du rapport
\newcommand{\hmwkShortTitle}{Méchant Virus} % Objet du rapport (version courte)
\newcommand{\hmwkAuthorName}{Inspecteur Gadget} % Vore nom
\newcommand{\hmwkCommanditaire}{Général Bob Baroud, officier de police judiciaire en résidence aux Kerguelen} % Nom de la personne requerante


%----------------------------------------------------------------------------------------
%	TITLE PAGE
%----------------------------------------------------------------------------------------

\title{
\centering{\includegraphics[width=.3\textwidth]{GN_logo.jpg}}\\
\vspace{2in}
\textmd{\textbf{\hmwkClass:\ \hmwkTitle}}\\
\normalsize\vspace{0.1in}\small{\hmwkDueDate}\\
\vspace{0.1in}\large{\textit{\hmwkClassInstructor\ \hmwkClassTime}}
\vspace{3in}
}

\author{\textbf{\hmwkAuthorName}}
\date{} % Insert date here if you want it to appear below your name

%----------------------------------------------------------------------------------------

\begin{document}

\maketitle

%----------------------------------------------------------------------------------------
%	TABLE OF CONTENTS
%----------------------------------------------------------------------------------------

%\setcounter{tocdepth}{1} % Uncomment this line if you don't want subsections listed in the ToC

\newpage
\tableofcontents
\newpage

%----------------------------------------------------------------------------------------
%	Saisine, Mission, Contexte
% Rappel de la saisine
% Qualification des infractions
% Contexte dans le cadre d'un PV
% Prestation de serment
%----------------------------------------------------------------------------------------

% To have just one problem per page, simply put a \clearpage after each problem

\begin{homeworkProblem}[Saisine, Mission, Contexte]
  \subsection*{Référence}
  Procès-verbal de Réquisition à personne qualifiée N$^o$ XXXXX/XXXXX/201X en date du 30 Octobre 201X, du \hmwkCommanditaire.
  \subsection*{Mission}
  Nous soussignés Inspecteur Gadget, etc.

  Désigné pour effectuer l'examen technique par le \hmwkCommanditaire, en sa réquisition du 30 Octobre 201X, comportant les missions suivantes:
  \begin{itemize}
  \item Procéder à l'examen du disque dure remis sous scellés dans le but de rechercher tout virus, etc.
  \end{itemize}

  Serment préalablement prêté de bien et fidèlement remplir notre mission en notre honneur et conscience, rapportons les opérations effectuées.
\end{homeworkProblem}

%----------------------------------------------------------------------------------------
%	Prise en compte du scellé
%----------------------------------------------------------------------------------------

\begin{homeworkProblem}[Prise en compte du scellé]



Le tant et tant à 11 h 06 prenons en compte le scellé N$^o$X, pièce N$^o$XX remis par le \hmwkCommanditaire (\autoref{fig:scelle}). etc.
  \begin{figure}
    \begin{center}
      \includegraphics[width=0.75\columnwidth]{scelle.jpg} % Image CC BY SA https://commons.wikimedia.org/wiki/File:CG_NY_scell%C3%A9.jpg?uselang=fr 
    \end{center}
    \caption{Le scellé}
    \label{fig:scelle}
  \end{figure}

\end{homeworkProblem}

%----------------------------------------------------------------------------------------
%	Protocole d'investifations
%----------------------------------------------------------------------------------------

\begin{homeworkProblem}[Protocole d'investigations]
  \subsection*{Détection de l'infection}

  Bien souvent, on a besoin d'intégrer du code dans le rapport, on peut faire ça comme çà :

  \pythonscript{Coucou}{Hello World}
  Et on peut afficher la réponse comme ceci :
\begin{lstlisting}[breaklines=true]
Coucou de la part du C3N !
\end{lstlisting}

\subsection*{Les subsections peuvent structurer les sections}
\lipsum[1]


\end{homeworkProblem}
%----------------------------------------------------------------------------------------
%	Résultat des investigations
%----------------------------------------------------------------------------------------

\begin{homeworkProblem}[Résultat des investigations]
  \lipsum[2]
\end{homeworkProblem}

%----------------------------------------------------------------------------------------
%	Discussion
% Explications en terms simples, pour le non spécialiste, de ce que signifient les résultats
%----------------------------------------------------------------------------------------

\begin{homeworkProblem}[Discussion]
  \lipsum[2]
\end{homeworkProblem}


%----------------------------------------------------------------------------------------
%	Conclusion
% Réponse aux questions posées par le requerant
%----------------------------------------------------------------------------------------

\begin{homeworkProblem}[Conclusion]
\lipsum[3]

Rapport Clos aux Kerguelen le \today.
\end{homeworkProblem}


\end{document}
